\documentclass[paper=a4, oneside, fontsize=14pt, titlepage]{article}
\usepackage[italian]{babel}
\usepackage[utf8x]{inputenc}
\usepackage{newlfont}
\usepackage{makeidx}
\usepackage{hyperref}										%Url visualization
\usepackage[edges]{forest}											%Filesystem tree

\usepackage{fancyhdr}										%Header customization
\pagestyle{fancy}
\lhead{Relazione al progetto}
\rfoot{Pagina \thepage}
\fancyfoot{} % clear all footer fields
\fancyfoot[LE,RO]{Pagina \thepage}           % page number in "outer" position of footer line

\usepackage{tikz}
\usetikzlibrary{trees}
\usetikzlibrary{positioning}
%%%%%%%%%%%% Tree folder %%%%%%%%%%%%%%%%%%%%%%%%%%%%%%%%%%%%%%%%

\definecolor{foldercolor}{RGB}{124,166,198}

\tikzset{pics/folder/.style={code={%
			\node[inner sep=0pt, minimum size=#1](-foldericon){};
			\node[folder style, inner sep=0pt, minimum width=0.3*#1, minimum height=0.6*#1, above right, xshift=0.05*#1] at (-foldericon.west){};
			\node[folder style, inner sep=0pt, minimum size=#1] at (-foldericon.center){};}
	},
	pics/folder/.default={20pt},
	folder style/.style={draw=foldercolor!80!black,top color=foldercolor!40,bottom color=foldercolor}
}

\forestset{is file/.style={edge path'/.expanded={%
			([xshift=\forestregister{folder indent}]!u.parent anchor) |- (.child anchor)},
		inner sep=1pt},
	this folder size/.style={edge path'/.expanded={%
			([xshift=\forestregister{folder indent}]!u.parent anchor) |- (.child anchor) pic[solid]{folder=#1}}, inner ysep=0.6*#1},
	folder tree indent/.style={before computing xy={l=#1}},
	folder icons/.style={folder, this folder size=#1, folder tree indent=3*#1},
	folder icons/.default={12pt},
}
%%%%%%%%%%%%%%%%%%%%%%%%%%%%%%%%%%%%%%%%%%%%%%%%%%%%%%%%%%%%%%%%

\begin{document}
	\begin{titlepage}
		\begin{center}
			{{\Large{\textsc{Università degli Studi di Roma Tor Vergata}}}} \rule[0.1cm]{5.8cm}{0.5mm}\\
			{\small{\bf DIPARTIMENTO DI INGEGNERIA\\Corso di Laurea in Ingegneria Informatica}}
		\end{center}
		\vspace{15mm}
		\begin{center}
			{\LARGE{\bf\textit{ Content Adaption}:}}\\
			\vspace{3mm}
			{\LARGE{\bf Adattamento dinamico di contenuti statici}}\\
			\vspace{19mm} {\large{\bf Relazione al progetto di Ingegneria di Internet e del Web}}
		\end{center}
		\vspace{10mm}
		\begin{center}
			\includegraphics[scale=0.3]{./images/world.jpg}
		\end{center}
		\vspace{10mm}
		\par
		\noindent
		\begin{minipage}[t]{0.47\textwidth}
			{\large{\bf Relatore:\\
					Chiar.mo Prof.\\
					Francesco Lo Presti}}
		\end{minipage}
		\hfill
		\begin{minipage}[t]{0.47\textwidth}\raggedleft
			{\large{\bf Presentata da:\\
					Federico Amici}}
		\end{minipage}
		\vspace{20mm}
		\begin{center}
			{\large{\bf $\sharp$ Sessione\\
					2017/2018 }}
		\end{center}
	\end{titlepage}
	\tableofcontents
	\newpage
	
	\section{Introduzione}
		\begin{flushleft}
			Questa relazione ha l'intento di illustrare, con la maggior chiarezza possibile, quanto è stato realizzato nell'ambito dello sviluppo di questo progetto relativo al corso di Ingegneria di Internet e del Web. Verranno illustrate altresì le scelte architetturali adottate nella realizzazione del sistema proposto, chiarendone i motivi alla base. All'interno della cartella principale del cd, è presente inoltre una presentazione contenente una raccolta di appunti stesi durante la realizzazione del progetto.
		\end{flushleft}
		\begin{center}
			\begin{tikzpicture}
			\draw (0,0) -- (2,0);
			\end{tikzpicture}
			\hspace{1cm}
			\begin{tikzpicture}
			\draw (0,0) -- (2,0);
			\end{tikzpicture}
		\end{center}
		\begin{flushleft}
			Durante qualsiasi comunicazione all'interno della rete \textit{Internet} troviamo un'entità che richiede una risorsa, sia essa una pagina HTML, un'immagine JPEG o un frammento di un film che si sta guardando in streaming. Il numero di entità che possono fornire la risorsa varia tra \textit{1} ed un arbitrario \textit{n} a seconda dell'architettura adottata per la realizzazione dell'applicazione che si sta utilizzando per richiedere la risorsa:
		\end{flushleft}
		\begin{itemize}
			\item se l'applicazione è stata realizzata secondo l'architettura \textit{client-server} sarà una sola entità a fornire la risorsa richiesta, ovvero il server stesso
			\item nel caso in cui l'applicazione sia stata progettata secondo l'architettura \textit{peer-to-peer}, la risorsa verrà fornita in generale da n entità che collaborano a tale scopo. L'elemento server centrale non viene necessariamente rimosso in quanto potrebbe risultare utile nella coordinazione fra i peer.
		\end{itemize}
		\vspace{5mm}
		\begin{center}
			\begin{figure}[ht!]
				\begin{tikzpicture}
					%Styles for the nodes
					[client/.style={circle, draw=blue=!80, fill=blue!30}]
					
					\node (server) at (0,0) [circle, draw=green!80, fill=green!30, thick] {Server};
					\node[client] (client1) [above right = of server] {client};
					\node[client] (client2) [above left = of server] {client};
					\node[client] (client3) [below right = of server] {client};
					\node[client] (client4) [below left = of server] {client};
					
					\draw [<->, very thick] (server.north east) -- (client1.south);
					\draw [<->, very thick] (server.north west) -- (client2.south);
					\draw [<->, very thick] (server.south east) -- (client3.north);
					\draw [<->, very thick] (server.south west) -- (client4.north);
				\end{tikzpicture}
				\hspace{2cm}
				\begin{tikzpicture}
					%Styles for the nodes
					[client/.style={circle, draw=blue=!80, fill=blue!30, thick}]
					
					\node (server) at (0,0) [circle, draw=green!80, fill=green!30, thick] {Server};
					\node[client] (client1) [above right = of server] {peer};
					\node[client] (client2) [above left = of server] {peer};
					\node[client] (client3) [below right = of server] {peer};
					\node[client] (client4) [below left = of server] {peer};
					
					\draw [<->, very thick] (server.north east) -- (client1.south);
					\draw [<->, very thick] (server.north west) -- (client2.south);
					\draw [<->, very thick] (server.south east) -- (client3.north);
					\draw [<->, very thick] (server.south west) -- (client4.north);
					\draw [<->, very thick] (client1.west) -- (client2.east);
					\draw [<->, very thick] (client1.south) -- (client3.north);
					\draw [<->, very thick] (client3.west) -- (client4.east);
					\draw [<->, very thick] (client4.north) -- (client2.south);
				\end{tikzpicture}
				\caption{Architetture client-server e peer-to-peer}
			\end{figure}

		\end{center}
		\vspace{1cm}
		\begin{flushleft}
			Nel caso di questo progetto, ci si riferirà chiaramente all'architettura \textit{client-server}.\newline
			
			Il processo di sviluppo che è stato adottato per la realizzazione di questo progetto prende le mosse dai processi iterativi ed incrementali, un esempio su tutti è lo \textit{Unified Process}. Un processo di sviluppo è \textit{iterativo} se, terminato il flusso di lavoro relativo ad una fase, è possibile ritornare all'inizio del flusso per correggere gli errori precedenti, ad esempio dovuti ad interpretazioni non corrette dei requisiti. \'{E} \textit{incrementale} dal momento che ad ogni \textit{iterazione} è possibile ampliare il progetto implementando nuovi requisiti, o semplicemente aggiungendoli all'attenzione dei progettisti. Senza entrare nel dettaglio di un processo di sviluppo particolare, si sottolinea che ognuno di essi ha una fase di inizializzazione più o meno complessa.\newline
			
			\'{E} da sottolineare il fatto che ogni \textit{iterazione} restituisce una versione del sistema \textit{parzialmente completa} che costituisce un punto di partenza per l'iterazione successiva. Dunque non si riparte mai da zero, a meno di gravi errori commessi nelle fasi precedenti.\newline
			
			La fase di \textit{\textbf{inizializzazione}} relativa a questo progetto è stata particolarmente lunga e laboriosa in quanto mancava una buona padronanza dell'argomento, ma soprattutto della maggior parte delle insidie in cui ci si può imbattere nel linguaggio C, la manipolazione delle stringhe un esempio su tutti.\newline
			Di conseguenza, la fase di inizializzazione è stata sfruttata principalmente per prendere dimestichezza sia con l'argomento sia con il linguaggio. \'{E} stato realizzato uno scheletro del programma affinché si potesse avere una base da ampliare e migliorare.\newline
			
			In definitiva, la fase di inizializzazione ha scorso i requisiti funzionali del sistema da realizzare e ne ha implementato le funzionalità fondamentali. Questo approccio ha avuto il vantaggio di restituire un eseguibile di base del sistema, nonché del codice che con probabilità molto alta sarebbe stato riutilizzato come base per ampliare le funzionalità già implementate.\newline
			
			Fatta salva questa relazione, non è stata approfondita la parte di documentazione, di notevole importanza in un processo di sviluppo come lo \textit{Unified Process} o la sua versione proprietaria di \textit{Rational} RUP.	
		\end{flushleft}
		
		\subsection{Installazione del programma}
			\begin{flushleft}
				Di seguito viene riportato il contenuto del CD contenente sia i sorgenti del programma realizzato sia il sorgente di questa stessa relazione. Il codice utilizzato per disegnare l'albero delle directory è stato reperito 
				\href{https://tex.stackexchange.com/questions/5073/making-a-simple-directory-tree}{qui}.
			\end{flushleft}
			\begin{center}
				\begin{figure}
					\scalebox{0.8}{
						\begin{forest}
							for tree={font=\sffamily, %grow'=0,
								folder indent=.9em, folder icons,
								edge=densely dotted}
							[CD
							[ImageApp, this folder size=20pt]
							[Relazione, this folder size=20pt]
							[ImageMagick-master, this folder size=20pt]
							[src, this folder size=20pt]
							[test, this folder size=20pt]
							[Unused code, this folder size=15pt]
							[install.sh, is file]
							[Presentazione.odp, is file]
							[Relazione.pdf, is file]
							]
						\end{forest}
					}
				\caption{Contenuto del CD}
				\end{figure}
			\end{center}
			\begin{flushleft}
				La procedura di installazione è stata riportata esaurientemente anche nella presentazione \textit{Presentazione.odp} affinché il lettore potesse avere ugualmente un supporto grafico immediato nella fase di installazione anche se non interessato a questa relazione.\newline
				
				Come il lettore avrà modo di constatare durante la fase di installazione, vengono richiesti i privilegi di amministratore, il che costituirebbe una violazione ai requisiti imposti dalla traccia. Tuttavia i privilegi vengono sfruttati solo in questa procedura affinché il programma funzioni correttamente. In particolare vengono utilizzati per installare le librerie relative alla trasformazione delle immagini \textit{ImageMagick}, all'accesso al database \textit{MySQL}, oltre all'installazione di tutti quei programmi di contorno che si è assunto il lettore non avesse installati sulla propria macchina, un esempio su tutti il server \textit{mysql-server}.\newline
				
				Aprendo il terminale nella cartella principale, è possibile cominciare l'installazione eseguendo il comando \textit{sh ./install.sh}, dove la chiamata di \textit{sh} può essere evitata rendendo il file eseguibile tramite \textit{chmod +x ./install.sh}.
				
				L'installazione del server \textit{MySQL} viene effettuata durante la precedente procedura, così come la sua popolazione. \'{E} possibile scegliere tra l'importazione del file già esportato oppure, vista la natura didattica del progetto, eseguire il programma che è stato realizzato per effettuare gli inserimenti nel database a partire dal file \textit{wurfl.xml}. Per completezza, nella cartella \textit{Unused code} è presente una versione di questo file che utilizza la \textit{CLI} del DBMS \textit{Postgresql}, dal momento che in prima battuta era stato scelto proprio questo per effettuare l'accesso al database. Avendo riscontrato tuttavia delle problematiche nella sottoposizione delle query al database, ed in particolare nella sottoposizione di query dinamiche, si è optato per il DBMS \textit{MySQL} che non ha manifestato tali asperità.
				
				Si ricorda anche in questo documento che per il corretto funzionamento del server è necessario, in fase di installazione del server \textit{mysql-server}, scegliere la password suggerita, per semplicità impostata a \textit{0000}. Si è adottata questa scelta poiché non è possibile recuperare la password inserita a tempo di installazione, o almeno non in maniera agevole, ed anche la sua reimpostazione non risulterebbe di grande aiuto. 
				
			\end{flushleft}
%%%%%%%%%%%%%%%%%%%%%%%%%%%%%%% Traccia del progetto %%%%%%%%%%%%%%%%%%%%%%%%%%%%%%%%%%%%%%
	\section{Traccia del progetto}
	\begin{flushleft}
		Lo scopo del progetto è quello di realizzare in linguaggio C usando l'API del socket di Berkeley un Web	server concorrente in grado di adattare dinamicamente il contenuto statico fornito in base alle caratteristiche del dispositivo del client. Il server dovrà fornire:
		\begin{enumerate}
			\item le funzionalità di base di un Web server (incluso il logging);
			\item il supporto minimale del protocollo HTTP/1.1 (connessioni persistenti, metodi \textit{GET} e \textit{HEAD});
			\item l'adattamento dinamico (on the fly) di immagini. Tale adattamento dovrà essere effettuato almeno per il formato \textit{JPEG} sulla base del fattore di qualità specificato dal client nell’header	\textit{Accept}; ad esempio, nel caso di \textit{Accept: image/jpeg; q=0.5, text/plain} il server fornirà in risposta l’immagine \textit{JPEG} richiesta con fattore di qualità pari a 0.5. L’adattamento potrà anche essere effettuato per altri formati di immagini ed applicando operazioni di manipolazione dell’immagine (ad es. conversione del formato, resizing, riduzione del numero di colori) sulla base di informazioni riguardanti il browser e contenute nell’header \textit{User-Agent} (si consiglia di usare \textit{WURFL} per ottenere le informazioni sulle caratteristiche dei dispositivi (\url{http://wurfl.sourceforge.net/}));
			\item il caching della versione risultante dall’operazione di adattamento, allo scopo di ridurre l’overhead computazionale dell’adattamento. Implementare il caching in modo da gestire la presenza di versioni multiple di una stessa risorsa.
		\end{enumerate}
		Per l’adattamento delle immagini usare un programma esterno, ad esempio:
		\begin{itemize}
			\item \textit{ImageMagick} (in particolare, il programma \textit{convert}), disponibile all’URL \url{http://www.imagemagick.org/};
			\item la libreria \textit{Independent JPEG Group} (in particolare, i programmi \textit{djpeg} e \textit{cjpeg}), disponibile all’URL \url{http://www.ijg.org/}.
		\end{itemize}
		Il server dovrà essere eseguito nello spazio utente ed essere in ascolto su una porta di default (configurabile); potrà essere testato tramite un browser reale o tramite un tool a riga di comando quale \textit{cURL}, disponibile all’URL \url{http://curl.haxx.se/}.
		Si richiede di valutare le prestazioni del Web server realizzato (non includendo le funzionalità  di adattamento dei contenuti) in termini di tempo di risposta e throughput e confrontarle con quelle di \textit{Apache 2.x} usando un semplice workload di tipo statico. La valutazione delle prestazioni può essere effettuata utilizzando il tool di benchmarking \textit{httperf} (disponibile all’URL \url{http://www.hpl.hp.com/personal/David\_Mosberger/httperf.html}).
	\end{flushleft}
	
	\section{Environment}
	\begin{flushleft}
		-WURFL: prima ipotesi, comunicazione tra C e JAVA. Ricerca più accurata: API C
		
		http://libtins.github.io/download/\#download
		
		https://askubuntu.com/questions/522372/installing-32-bit-libraries-on-ubuntu-14-04-lts-64-bit
		
		https://askubuntu.com/questions/864188/pcap-library-not-found
		
		https://gist.github.com/rodleviton/74e22e952bd6e7e5bee1
		
		http://www.cs.colby.edu/maxwell/courses/tutorials/maketutor/
		
		https://www.youtube.com/watch?v=q3c6O85\_LoA
		
		Clion utilizzato solo per debug di piccole unita di test (linking difficoltoso)
		
		ECPG install	https://apps.ubuntu.com/cat/applications/precise/libecpg-dev/
		
		ERRORE NELL'XML (guarda screenshot)
		
		INSTALLAZIONE libpq-dev
		
		DA Ecpg a libpq per gestire le query dinamiche
		
		Cmake non compila ma senza lipbq in cmakelist non parte
		
		Sia ID che UserAgent sono chiavi primarie
		
		Dimensione della foto inviata nell'header Content length
		
		log in xml
		
		http://www.thegeekstuff.com/2010/04/inotify-c-program-example/
		
		https://github.com/nextcloud/client\_theming/issues/25
		
		https://www.postgresql.org/docs/8.1/static/libpq-threading.html
		
		http://www.lightbox.ca/pg2mysql.php
		
		Utilizzo valgrind
		
		http://www.tldp.org/HOWTO/html\_single/TCP-Keepalive-HOWTO/
		
		Python o LUA per GUI
		
		http://www.it.uom.gr/teaching/c\_sys/node30.html
		
		http://www.linuxbrigade.com/reduce-time\_wait-socket-connections/
		
		http://lkml.iu.edu/hypermail/linux/kernel/0106.1/1154.html
		
		http://www.texample.net/tikz/examples/filesystem-tree/
	\end{flushleft}
	
	\section{Strutture principali}
		\subsection{\textit{struct Server}}
	\begin{flushleft}
		La \textit{struct Request} si occupa di raccogliere le informazioni relative alla richiesta del client. I suoi campi sono:
		\begin{itemize}
			\item \textit{image\_name}: \'{e} il nome del file richiesto dal client
		\end{itemize}
	\end{flushleft}
	\subsection{\textit{struct Request}}
	\begin{flushleft}
		La \textit{struct Request} si occupa di raccogliere le informazioni relative alla richiesta del client. I suoi campi sono:
		\begin{itemize}
			\item \textit{image\_name}: \'{e} il nome del file richiesto dal client
		\end{itemize}
	\end{flushleft}
	
	\section{Requisiti non funzionali}
	\begin{flushleft}
		-I nomi delle immagini devono essere lunghi almeno tre lettere, ed al massimo 32

		-l'estensione determina univocamente il file richiesto, il formato richiesto viene recuperato dall'header content type
	\end{flushleft}
	\section{Punti di estensione}
	\begin{flushleft}
		Uno dei punti del programma che potrebbe essere approfondito è il dimensionamento del pool in base al carico di richieste in arrivo al sistema. Per come è stato realizzato il sistema, oltre allo sviluppo dell'algoritmo che si occupa della gestione delle dimensioni del pool, uno dei problemi di cui si dovrebbe tener conto è la gestione del numero di connessioni alla base di dati.\newline
		
		Altro punto che potrebbe essere approfondito è lo sviluppo di un algoritmo che scelta in maniera maggiormente oculata la dimensione del pool di thread dinamicamente a tempo di esecuzione.
	\end{flushleft}	
	\section{Annotazioni}
	\begin{flushleft}
		-non e stato mantenuto l'array di fd perche se la connessione dura 5 secondi, e improbabile che il client richieda due volte la stessa immagine
		-ifndef E' risultato utile per includere l'header Basic in più sorgenti, fatto che altrimenti avrebbe generato errori a tempo di compilazione dato che si stava tentando di ridefinire delle struct.\newline
		-import del database effettuata con strumenti esterni: anche se non prestante non viene effettuato a runtime
		-Nè Facebook nè google supportano keep-alive
		-in sede di chiusura della connessione del client, viene settato a 0 solo connsd in modo da sfruttare il già inizializzato mutex
		-RICONOSCERE MESSAGGI SPURI
		-select si blocca perchè l'fd viene temporaneamente rimosso dall'insieme dei controllabili, generando tuttavia deadlock. Alternative? forse poll()
		-cache query sia per prestazioni sia per evitare errori
		-processo di sviluppo: "Albero" poi iterativo
		-preferibile allocazione sullo stack poichè le allocazioni sull'heap sono in mutua esclusione fra i thread (uno per volta) (e poi l'heap è unico per il processo)
		-mysql connessioni https://dba.stackexchange.com/questions/47131/how-to-get-rid-of-maximum-user-connections-error
		
		-user agent 0 0 IOT (cavolata?)
		-lista dei file in cache salvata in un file e mappata in memoria -> crash del server preserva la lista (almeno la maggior parte)
		
		-dimensione del file di cache limitata a 64KB per evitare privilegi di amministratore (massima quantità di memoria bloccata pari a 64KB come da foto / Bisognerebbe controllare)
		-favicon.ico creata con convert (si potrebbe fare a runtime, ma chi mo fa fa?)
		
		-installare mysql server
		-creare utente root con password
		
		-libxml2-dev
		
		-per imagemagick: download sorgente, sudo apt-get install libmagickwand-dev, sudo apt-get install libmagickcore-dev, ./configure with jpg png e modules nella cartella, make, sudo make install, sudo ldconfig /usr/local/lib, installare brew con linuxbrew-wrapper, installare brew install imagemagick


		-assunzione: le estensioni richieste hanno 3 lettere, se non riesco a leggere l'immagine tramite chiamata MagickReadImage(), l'immagine non e in memoria

		/* OPERATION NOT PERMITTED
		struct rlimit rlim;
		rlim.rlim\_cur = 524288;
		rlim.rlim\_max = 524288;

		errno = 0;
		if(setrlimit(RLIMIT\_MEMLOCK, \&rlim) == -1)
		fprintf(stderr, "%s\n", strerror(errno));

		getrlimit(RLIMIT\_MEMLOCK, \&rlim);
		printf("max mem %ld %ld \n", rlim.rlim_cur, rlim.rlim_max);
		*/
	\end{flushleft}
	\section{Test}
	\begin{flushleft}
		
	\end{flushleft}
	
\end{document}
\grid
 
\grid
\grid
\grid
